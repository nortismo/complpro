% !TeX spellcheck = en_GB
\hypertarget{intro}{%
\section{Intro}\label{intro}}

A complex system is hard to simplify or cannot be simplified at all.

\hypertarget{complicated-vs.complex-system}{%
\subsection{Complicated vs. complex
system}\label{complicated-vs.complex-system}}

A \textbf{complex system} has a lot of elements. It's hard to explain
but not because of the amount of different elements but the
correlation between them. It's not possible to cut the system into subsystems
due to loops. It would change the behaviour of the system.

\emph{Example: Company} It's not possible to understand a company if you'll only know a part of it.

\mbox{}\\
If a system has many elements and is therefore not easy to explain, we
have a \textbf{complicated system}. A complicated system is
''reducible'' by cutting it into small systems (subsystems), which can
be explained.

\emph{Example: Car} It's a complicated system, but the engine can be analysed separately
(or even be splited into more subsystems).

\hypertarget{features-of-a-complex-system}{%
\subsection{Features of a complex
system}\label{features-of-a-complex-system}}

There are no typical features of a complex system, but it often shows
the following properties:

\begin{itemize}
\tightlist
\item High number of elements
\item Non-linear interactions between the elements
\item Delayed effects
\item Negative and positive feedback
\item Network-like structure
\item Open
\item Universal
\item Dynamic
\item Robust
\item Creative and innovative
\item Unpredictable
\item Differentiated sensitivity
\item Not monitorable
\end{itemize}

\hypertarget{models-and-simulation}{%
\subsection{Models and simulation}\label{models-and-simulation}}

To manage complexity it is required to build models. A model is a
simplified description of reality.\\
The \textbf{usefulness of a model} is, to test changes within a model and
transfer the gained knowledge to the real world. If this is not
possible, the model is \textbf{useless}.\\
A model must be as accurate as required, but details must be erased by a model.

\mbox{}\\
The usage of a model to determine how things are going to act in a
certain situation (in all likelihood), is called \textbf{simulation}.\\
A model can only be used for simulation it is intended to, it can not be generalized.
