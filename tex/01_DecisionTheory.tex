% !TeX spellcheck = en_GB
\section{Decision Theory}
There are two perspectives on the decision theory:

\begin{description}
	\item[Descriptive decision theory] Describes the way how decisions
	were made (already done).
	\item[Normative decision theory] Describes how decisions
	should be made. What options are available, what should be respected?
\end{description}

\subsection{Descriptive Decision Theory}
The descriptive decision theory can be splited into multiple types of decissions:

\begin{description}
	\item[Rational decision] Rational decisions are	\textbf{targeted
	throughout, goal oriented and based on objective informations}.
	The process while getting to a rational decision is a
	\textbf{systematic procedure and uses clear methodological rules}.
	This rules can be understood by \textbf{those not involved}.
	\item[Intuitive decision] The fundamental of such decisions is the
	\textbf{knowledge} you have and \textbf{the experience of everything}
	you ever experienced or of happenings and knowledge you earned from
	other people. Decisions are made of \textbf{gut feeling}.
\end{description}

\subsection{Mental Models}
A mental model is a \textbf{simplified mental description of
the experienced world} with the purpose of \textbf{handling the large
quantity of information} in order to be able to \textbf{make decisions
quickly}.

\mbox{}\\
Mental Models are Subconscious, gained by experience and they are adapted to
specific situations immediately.

\mbox{}\\
\emph{Learning is nothing else than adapting mental models.
If you are not able to adapt mental models, you are not able 
to make decisions quickly.}

\paragraph{Example}\mbox{}\\
If your coin doesn't get accepted in the vendor machine, you need
to rub it on the surface of the machine and try it again. And if this
works, it even strengthen your mental model, but actually this process
with the coin doesn't help in any way.

\mbox{}\\
Second example: You try to cross the road and you are at the edge of the
side walk. If there are no noise, you are able to cross the road. But no
noise is actually not a proof that the street if safe to cross.

\subsection{Dangers of mental models}

\emph{Our theories are our inventions, but they may be merely il-reasoned
	guesses, bold conjectures, hypotheses. Out of these we create a world: not
	the real world,	but our own nets which we try to catch the real world.}

By dealing with models, the human tends to believe the most complex
model is correct. It's essential to stick to the simplest available
model, unless it's proven wrong.

Prejustice is also a big issue (D.L. Rosehans experiment).

\subsubsection{Problems of mental models}

\begin{enumerate}
\tightlist
\item Superstitiousness (ticket machine on the bus)
\item Prejudices (Rosenhan experiment)
\item Mental model can generate their own momentum (selective search after
confirmation) (e.g. ''Filterbubble'', Conspiracy theories)
\item Conservatism: Mental models develop based on experience
\item False security: If mental models are not perceived as such they are
not questioned.
\end{enumerate}

\subsubsection{Handle mental models}

\textbf{Mental models cannot be verified!}

Defuse problems with the following steps:
\begin{enumerate}
\tightlist
\item Become familiar with your own mental models.
make mental models explicit.
\item Critical thinking
\item Exchanging information about mental models
\end{enumerate}
