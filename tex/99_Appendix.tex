% !TeX spellcheck = en_GB
\section{Appendix}

\subsection{D.L. Rosenhan experiment}
In 1973, D.L. Rosenhan investigated whether it was possible for staff in
psychiatric clinics to distinguish between people with a mental illness and
healthy people. To do this, he sneaked eight healthy people as supposed
patients into one clinic each. After their admission to the clinics, these people
behaved "normally". None of the eight patients was discovered as being a
fake patient and they had to spend an average of 19 days (a minimum of 7
days and a maximum of 52 days) in the institution until they were released.
In the next experiment, Rosenhan "warned" a clinic that within the next three
months one or several fake patients were to be brought in.
41 patients were detected as candidates by at least one staff member. A
psychiatrist even estimated that 19 patients were not really ill. However,
during this time period, no fake patients had actually been admitted.

\subsection{Causality vs.Correlation (MEP Question)}

\subsubsection{Correlation}
Two factors are correlated if a mathematical dependency between them can
be statistically proven.\\
\emph{Can be proven statistically!}

\subsubsection{Causality}
There is a casual relationship between two factors if a cause-effect
relationship exists between the two.\\
\emph{Can not be proven!}

\mbox{}\\
Correlation can be proven, strictly speaking causality
cannot.
